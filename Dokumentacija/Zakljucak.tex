\chapter{Zaključak i budući rad}
		
		Cilj ovog projekta bio je razviti web aplikaciju nazvanu \textit{Connectima} koja predstavlja platformu za promociju raznih vrsta događanja uz mogućnost iskazivanja interesa za događanja, recenziranje završenih događanja i informiranje o organizatorima događanja. Iz projektnog smo zadataka izlučili 
		funkcionalne, nefunkcionalne i druge zahtjeve, konceptualno osmisliti i zatim proveli implementaciju te cijeli proces i sve važne komponente razvijenog programskog rješenja detaljno dokumentirali. Projektni je cilj ostvaren nakon 13 tjedana rada. Rad na projektu dijelimo u dvije faze. 
		
		Početak prve faze označilo je okupljanje projektnog tima, uspostavljanje kanala komunikacije i podjela odgovornosti unutar tima. U ovoj je fazi naglasak bio na detaljnoj razradi zahtjeva,  definiranju arhitekture sustava i tehnologija koje će se koristiti za implementaciju. Od velike nam je važnosti bilo u ovoj fazi definirati željeni izgled i vizualni dizajn aplikacije. Jasna slika aplikacije koju izrađujemo postala je ključan temelj za daljnji napredak u projektu.
		
		Nakon što su u prvoj fazi ostvarene samo najosnovnije funkcije sustava, u drugoj je fazi stavljen naglasak na samu implementaciju prethodno definiranih obrazaca upotrebe. U ovoj ubrzanoj fazi projekta od iznimne nam je važnosti bila međusobna komunikacija i surađivanje. U tom se kontekstu aplikacija Notion pokazala kao iznimno koristan alat. Omogućila nam je sustavno praćenje popisa preostalih zadataka, a također je poslužila i kao platforma gdje su članovi tima međusobno mogli dijeliti upute, objašnjenja i korisne informacije.
		
		Razvijenu je aplikaciju moguće unaprijediti na mnogo načina. Integracija s e-mailom umjesto inboxa u aplikaciji, komentiranje nadolazećih događanja, opcija odgovaranja i reagiranja na recenzije, prikaz nadolazećih događanja u kalendaru te mogućnost sinkronizacije s popularnim kalendarima poput Google kalendara neka su poboljšanja koja smo razmatrali tijekom razvoja aplikacije. Budući bi rad mogao uključivati i izradu mobilne aplikacije kako bi razvijeni sustav mogao konkurirati vodećim platformama za promociju događanja. 
		
		Sudjelovanje u ovom projektu pokazalo se izuzetno korisnim iskustvom za sve članove tima. Tijekom rada na projektu, članovi tima usvojili su brojna nova znanja o tehnologijama poput Git-a i LaTeX-a i naučili koristiti suvremene radne okvire za izradu web aplikacija. Jednako važnim kao i nova tehnološka znanja smatramo i stečeno iskustvo rada u timu koje nas je naučilo da zajednički napor, istraživanje, učenje i suradnja oblikuju ne samo kôd, već i nas same. 
		
		\eject 